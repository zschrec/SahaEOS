
\documentclass{article}

\usepackage{amsmath}
\usepackage[margin=1in]{geometry}

\begin{document}

	
\begin{titlepage}

	\begin{center}

		~\\[10cm]
		\textsc{\Large Saha Equation of State} \\
		\textsc{Math Document} \\
		\textsc{Zachariah Schrecengost}

	\end{center}

\end{titlepage}

	\newpage

	\section{Saha Equation}
		The general Saha equation is given by
		\begin{equation} \label{eq:gen_saha}
			\frac{n^{r+1}_i}{n^r_i} = 
				\left( \frac{2 \pi m_e k}{h^2} \right)^\frac{3}{2} T^\frac{3}{2} 
				\frac{2 B^{r+1}_i exp\left(\frac{-\chi^r_i}{kT}\right)}{B^r_i N_e}
		\end{equation}
		where for element $i$ and ionization $r$. For ease of expression, we will take
		\begin{equation} \label{eq:saha_constant}
			C_i =	\left( \frac{2 \pi m_e k}{h^2} \right)^\frac{3}{2}  
				\frac{2 B^{r+1}_i}{B^r_i}
		\end{equation}
		So the Saha equation is re-expressed by
		\begin{equation} \label{eq:revised_saha}
			\frac{n^{r+1}_i}{n^r_i} = C_i T^\frac{3}{2} 
				exp\left(\frac{-\chi^r_i}{kT}\right) \frac{1}{N_e}
		\end{equation}
		This is the form we will manipulate throughout this document.

		\subsection{Iterative Method}
			Initially, the iterative method using the Saha equation was as follows:
			\begin{enumerate}
				\item Choose a temperature $T$ and a density $\rho$
				\item Guess an initial electron density $N_e$
				\item Solve $\frac{n^{r+1}_i}{n^r_i}$ for all ratios applicable to 
					element $i$. Hydrogen, for example, only has $\frac{n^{1}_H}{n^0_H}$
				\item Use the ratios to solve the ionization fraction, given by
					\begin{equation} \label{eq:ion_frac}
						y^r_i = \frac{n^r_i}{\displaystyle\sum^{z(i)}_{s=0}{n^s_i}}
					\end{equation}
				\item Use $y^r_i$ to solve for the average number of free electrons contributed 
					by element $i$ using
					\begin{equation} \label{eq:avg_elec}
						\nu_e(i) = \displaystyle\sum_{s=0}^{z(i)}{s y^s_i}
					\end{equation}
				\item Recompute the electron density as
					\begin{equation} \label{eq:new_Ne}
						N_{e1} = \rho N_o \displaystyle\sum_{i}{\frac{x_i}{A_i} \nu_e(i) } 
					\end{equation}
					where $N_o$ is Avogadro's number, $x_i$ is the grams of element $i$ per grams
					grams of mixuture, and $A_i$ is the standard atomic weight of element $i$.
			\end{enumerate}
			There steps are repeated until $N_e$ and $N_{e1}$ are within some tolerance.
			
			\subsubsection{Computing the ionization fraction}
				We must manipulate equation \ref{eq:ion_frac} because once we compute the Saha
				equation, we only have the ratios $\frac{n^{r+1}_i}{n^r_i}$. We will give examples
				of the new ionization fraction equations which is a bit more straightforward than
				just an equation. Also, we never need to compute the value $y^0_i$ because it 
				contributes zero electrons to the recalculation of the electron density.

				\paragraph{Hydrogen}
					For hydrogen, we only have the ratio $\frac{n^1}{n^0}$, where we have dropped
					the subscript denoting the element to reduce clutter. The only ionization 
					fraction we need to compute is then given by
					\begin{equation*} 
						y^1 = \frac{n^1}{n^0+n^1} = \frac{1}{\frac{n^0}{n^1}+1}
					\end{equation*}
					\begin{equation} 
						y^1 = \frac{1}{\frac{1}{\frac{n^1}{n^0}}+1}
					\end{equation}

				\paragraph{Helium} 
					For helium, we have the ratios $\frac{n^1}{n^0}$ and $\frac{n^2}{n^1}$ and 
					we have two ionization fraction we must compute. The first is given by
					\begin{equation*} 
						y^1 = \frac{n^1}{n^0+n^1+n^2} = 
							\frac{1}{\frac{n^0}{n^1}+1+\frac{n^2}{n^1}}
					\end{equation*}
					\begin{equation} 
						y^1 = \frac{1}{\frac{1}{\frac{n^1}{n^0}}+1+\frac{n^2}{n^1}}
					\end{equation}
					and the second is given by
					\begin{equation*} 
						y^2 = \frac{n^2}{n^0+n^1+n^2} = 
							\frac{1}{\frac{n^0}{n^2}+\frac{n^1}{n^2} + 1} =
							\frac{1}{\frac{1}{\frac{n^2}{n^0}} + \frac{1}{\frac{n^2}{n^1}} + 1}
					\end{equation*}
					\begin{equation} 
						y^2 = \frac{1}{\frac{1}{\frac{n^1}{n^0}\frac{n^2}{n^1}} + 
							\frac{1}{\frac{n^2}{n^1}} + 1}
					\end{equation}
				
				\paragraph{Carbon}
					TODO

			\subsubsection{Using the computed electron density}

						

	\section{Pressure}
		The general expression for pressure is given by
		\begin{equation}
			P = n k T
		\end{equation}
		where we have $n = n_i + N_e$ which includes the number of particles of a 
		given element $i$ and the number of free electrons. Therefore
		\begin{equation}
			P = (n_i + N_e) k T
		\end{equation}
		In the following sections, we will examine how the equation of pressure 
		changes when considering various elements.

	\section{Internal Energy}

	\section{Mixture of Pure Hydrogen}
		\subsection{The Saha Equation}
			The Saha equation for hydrogen is given by
			\begin{equation} \label{eq:h_saha}
				\frac{n^{1}_H}{n^0_H} = C_H T^\frac{3}{2} 
					exp\left(\frac{-\chi^0_H}{kT}\right) \frac{1}{N_e}
			\end{equation}
			where
			\begin{equation*} \label{eq:saha_constant_H}
				C_H = \left( \frac{2 \pi m_e k}{h^2} \right)^\frac{3}{2}  
					\frac{2 B^{1}_H}{B^0_H}
			\end{equation*}


		\subsection{Pressure}
			The pressure of an ionized mixture of pure hydrogen is given by
			\begin{equation}
				P = (n_H + N_e) k T
			\end{equation}
			and we can express the number of hydrogen atoms as
			\begin{equation}
				n_H = \rho N_o \frac{x_H}{A_H} 
			\end{equation}
			where $A_H = 1.00794$ and $x_H = 1$.

			\subsubsection{Manipulation of the Saha equation}
				\paragraph{Solving for temperature from pressure}
					\begin{equation*}
						P = (n_H + N_e) k T 
					\end{equation*}
					\begin{equation} \label{eq:temp_pressure_H}
						T = \frac{P}{(n_H + N_e) k}
					\end{equation}
					Now use the temperature in equation \ref{eq:temp_pressure_H} in 
					\begin{equation} \label{eq:saha_temp_pressure_H}
						\frac{n^{1}_H}{n^0_H} =  
						\left( \frac{2 \pi m_e k}{h^2} \right)^\frac{3}{2} \frac{2 B^{1}_H}{B^0_H}
						T^\frac{3}{2} exp\left(\frac{-\chi^0_H}{kT}\right) \frac{1}{N_e}
					\end{equation}
					We will now define a function $g(N_e)$ by substituting $T$ into 
					equation \ref{eq:saha_temp_pressure_H}
					\begin{equation}
						\frac{n^{1}_H}{n^0_H} = g(N_e) = C_H \left(\frac{P}{(n_H + N_e) k}\right)^\frac{3}{2}
						exp\left(\frac{-\chi^0_H}{k_{eV}\left(\frac{P}{(n_H + N_e) k}\right)}\right) 
						\frac{1}{N_e}
					\end{equation}
					Using $g(N_e)$, we can express the ionization fraction as
					\begin{equation} 
						y^1_H = \frac{1}{\frac{1}{g(N_e)} + 1}
					\end{equation}
					We then have
					\begin{equation}
						\nu_e(H) = 1*y^1_H = y^1_H
					\end{equation}
					Therefore, we want to solve
					\begin{equation} 
						N_{e} = n_H * \nu_e(H) = n_H * y^1_H = n_H * \frac{1}{\frac{1}{g(N_e)} + 1}
					\end{equation}
					\begin{equation} \label{eq:N_e_hydrogen_solve} 
						N_{e} = n_H * \frac{1}{\frac{1}{C_H \left(\frac{P}{(n_H + N_e) k}\right)^\frac{3}{2}
						exp\left(\frac{-\chi^0_H}{k_{eV}\left(\frac{P}{(n_H + N_e) k}\right)}\right) 
						\frac{1}{N_e}} + 1}
					\end{equation}
					It is clear that equation \ref{eq:N_e_hydrogen_solve} is nonlinear, however we should
					be able to use root finding numerical methods to solve it... should. We can use $G(N_e)$
					for this.
					\begin{equation} \label{eq:N_e_hydrogen_solve_zero} 
						G(N_e) = n_H * \frac{1}{\frac{1}{C_H \left(\frac{P}{(n_H + N_e) k}\right)^\frac{3}{2}
						exp\left(\frac{-\chi^0_H}{k_{eV}\left(\frac{P}{(n_H + N_e) k}\right)}\right) 
						\frac{1}{N_e}} + 1} - N_e = 0
					\end{equation}

%					\begin{equation*} 
%					\frac{n^{1}_H}{n^0_H} =  
%					\left( \frac{2 \pi m k}{h^2} \right)^\frac{3}{2} \frac{2 B^{1}_H}{B^0_H}
%					\left(\frac{P}{(n_H + N_e) k}\right)^\frac{3}{2} 
%					exp\left(\frac{-\chi^0_H}{k\frac{P}{(n_H + N_e) k}}\right) \frac{1}{N_e}
%				\end{equation*}
%					\begin{equation*} 
%						\frac{n^{1}_H}{n^0_H} =  
%						\left( \frac{2 \pi m}{h^2} \right)^\frac{3}{2} \frac{2 B^{1}_H}{B^0_H} 
%						P^\frac{3}{2} exp\left(\frac{-\chi^0_H}{kT}\right) 
%						\frac{1}{N_e (n_H+N_e)^\frac{3}{2}}
%					\end{equation*}
				\paragraph{Solving for $N_e$ from pressure}
					\begin{equation*}
						P = (n_H + N_e) k T \rightarrow
						\frac{P}{kT} = n_H + N_e
					\end{equation*}
					\begin{equation}
						N_e =  \frac{P}{kT} - n_H
					\end{equation}
					Also
					\begin{equation}
						k T N_e =  P - n_H k T
					\end{equation}

		\subsection{Internal Energy} 
			\subsubsection{Manipulation of the Saha equation}
				\paragraph{Solving for temperature from internal energy}

				\paragraph{Solving for $N_e$ from internal energy}


		

\end{document}
